\documentclass{article}

\usepackage[all]{xy}
\usepackage{microtype}
\usepackage{xcolor}

\title{Understanding Biabduction}
\author{Radu Grigore \and Rasmus Lerchedahl Petersen}

\newcommand{\note}[2]{\textcolor{teal}{[\textcolor{red}{#1}: #2]}}
\newcommand{\rg}[1]{\note{rg}{#1}}
\newcommand{\rlp}[1]{\note{rlp}{#1}}

\newcommand{\hoare}[3]{\{#1\}#2\{#3\}}
\newcommand{\set}[2]{\{#1 \mid #2\}}
\newcommand{\ts}{\vdash} % rg: What is ts?

\begin{document}
\maketitle
\section{Introduction}\label{sec:intro}
We shell attempt to give a clear account of the technique for symbolic
execution known as biabduction, suitable for a graduate student. We
will assume access to an abductive theorem prover, i.e. that we can
ask questions of the form $P * A^? \ts Q * F^?$, where $P$ and $Q$ are
known and $A$ (the abducted anti-frame) and $F$ (the abducted frame)
are synthesized. Thus the pair $(A, F)$ constitutes the answer to the
question.

How to program a theorem capable of answering such questions is
another interesting challenge which we might address at a later stage.
\section{Symbolic Execution}\label{sec:symbexe}
\subsection{Idea}\label{sec:symbexe:idea}
We maintain a symbolic state while (symbolically) executing each
statement. Statements are represented as Hoare triples, a pre- and a
post-condition.

For biabduction, the state consists of a pair of heaps $(H, L)$. The
semantics of this pair is that ``if $L$ is provided at the beginning,
then $H$ is available now''.

Thus, to execute $\hoare P C Q$ from the state $(H, L)$, we first ask
the abductive question $H * A^? \ts P * F^?$. If we get back the answer
$(A, F)$, we know that adding $A$ to the current heap will allow us to
execute $C$. We cannot add anything to the current heap, but we could
have added more than $L$ in the beginning. The frame rule will then
ensure, that it is available now. \rg{I suppose that we have to check
the side condition of the frame rule. Right?}

In that case, we can execute $C$ from $P * F$ and use the frame rule
again to conclude that the resulting state will be $Q * F$.

So after the staement $C$, we have the symbolic state $(Q * F, L *
A)$, because if $L * A$ is provided in the beginning, then $H * A$ is
available to $C$ which will then execute to reach $Q * F$.

As a diagram:
\[
\xymatrix{
{\vdots} \ar[d]^{(H, L)} \\
*+[F]{\hoare P C Q} \ar[d]^{(Q * F, L * A)} &
{\mathrm{when}\ H * A \ts P * F} \\
{\vdots}
}
\]
When we reach the final statement, we have obtained a specification
for the procedure. Because we know that if $L$ is provided in the
beginning then $H$ will be available in the end. Thus the procedure~$C$
satisfies the Hoare triple $\hoare L C H$.

\subsection{Base case}\label{sec:symbexe:base}
The base case is slightly more complicated than the idea described
above, because each statement can potentially be reached in more than
one way. Similarly, in the case of a branching statement, it could
have multiple continuations.

In general we have a control flow graph where each node is a
statement. It can have multiple incoming flows and multiple outgoing
flows. Also each flow is in general a set of possible states (pairs of
heaps).
\[
\xymatrix{
{\vdots} \ar[dr]^{S_1} & {\vdots} \ar[d]_{\ldots}^{\ldots} & {\vdots} \ar[dl]_{S_n} \\
& *+[F]{\hoare P C Q} \ar[dl]^{S} \ar[d]_{\ldots}^{\ldots} \ar[dr]_{S}
& & S = \set{(Q * F, L * A)}{\exists i, H.\ (H, L)\in S_i\ \land\ H * A \ts P * F} \\
{\vdots} & {\vdots} & {\vdots}
}
\]

\subsection{Loops}\label{sec:symbexe:loops}
Loops require a fixed point calculation. We can detect back edges via
a depth first search and insert abstraction nodes.

\subsection{Abstraction}\label{sec:symbexe:abstraction}
For each abstraction node, we over approximate the inflows by the single outflow
\[
\xymatrix{
{\vdots} \ar[d]^{S} & {S = \{(H_1, L_1), \ldots, (H_n, L_n)\}}\\
*+[F]{\hoare P C Q} \ar[d]^{\{(Q * F, L * A)\}} & H_1 \lor \ldots \lor H_n \ts H \\
{\vdots} & L \ts L_1 \land \ldots \land L_n
}
\]
Then we repeat until we reach a fixed point.
\subsection{Recursion}\label{sec:symbexe:recursion}
Recursion requires another fixed point computation.
\end{document}
